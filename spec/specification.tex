\documentclass[11pt,a4paper]{report}

\usepackage{amsmath,amsfonts,amssymb,hyperref,svg}

\title{
    \centering{\includesvg{../icon}}\\
    Specification of the On-Disk Format of The TFS File System
}
\author{TFS team}
\date{\today}

% Constants
\newcommand{\clustersize}{512 }
\newcommand{\minimumsectorsize}{512 }
\newcommand{\versionnumber}{0 }

\begin{document}
    \maketitle

    \begin{abstract}
        We give a complete specification of the on-disk representation of the
        TFS file system. Certain implementation details are covered.
    \end{abstract}

    \tableofcontents

    \chapter{Introduction}
    This specification should detail the TFS file system such that it can be
    implemented without ambiguity. As such, it should bridge various
    implementation under one united on-disk format.

    It will not, however, provide the algorithms used to manipulate this format
    efficiently. It will only specify it's static, disk format.

    \section{Assumptions and guarantees}
    \label{assumptions_guarantees}
        TFS provides following guarantees:

        \begin{itemize}
            \item Unless data corruption happens, the disk should never be in
                an inconsistent state\footnote{TFS achieves this without using
                journaling or a transactional model.}. Poweroff and the alike
                should not affect the system such that it enters an invalid or
                inconsistent state.
        \end{itemize}

        Provided that following premises hold:

        \begin{itemize}
            \item Any sector (assumed to be a power of two of at least
                \minimumsectorsize bytes) can be read and written atomically,
                i.e.\ it is never partially written, and interrupting the write
                will never render the disk in a state in which it not already
                is written or retaining the old data.
        \end{itemize}

        Data corruption can break these guarantees or premises, and TFS
        encompasses certain measures against such corruption, but they are
        strictly speaking heuristic, like any error detection and correction
        method, as the damage could be across all the disks.

    \chapter{Disk header}
    \label{header}
    The first \clustersize bytes are reserved for the ``disk header'' which
    contains unencrypted configuration and information about the state.

    \section{Introducer (byte 0-16)}
        \subsection{Magic number (byte 0-8)}
        The first 8 bytes are reserved for a magic number, which is used for
        determining if it does indeed store TFS\@. It is specified to store the
        string ``\texttt{TFS fmt }'' (note the space) in ASCII.

        If the format does only partially conform to this specification, the
        string ``\texttt{\textasciitilde TFS fmt}'' is used instead. Partially
        compatible implementations may use checksum algorithm, vdevs,
        compression algorithms, and so on not specified in this document, but
        they should abide the same format.

        \subsection{Version number (byte 8-12)}
        \label{header:versionnumber}
        This field stores a version number, in little-endian. By this revision,
        said number is \versionnumber.

        Breaking changes will increment the higher half of this number.

    \section{Identification (byte 16-32)}
        \subsection{Unique ID (byte 16-32)}
        \label{uid}
        This is an 128-bit little-endian number, chosen at random. It should
        never be shared with a third-party.

    \section{Configuration (byte 32-48)}
        \subsection{Checksum algorithm (byte 32-34)}
        \label{config:checksum}
        This field stores a number in little-endian defining checksum algorithm
        in use.

        \begin{description}
            \item [$1$] The SeaHash algorithm as described
                in~\ref{algorithm:seahash}
            \item [$\geq 2^{15}$] Implementation defined.
        \end{description}

    \section{State (byte 48-64)}
        \subsection{State flag (byte 48)}
        \label{header:consistency}
        This little-endian integer takes one of the following
        values\footnote{The reason it takes two bytes instead of one is to be
        able to detect errors.}:

        \begin{description}
            \item [$\texttt{00}_{16}$] Disk I/O stream was properly closed.
            \item [$\texttt{01}_{16}$] Disk I/O stream was not properly
                closed\footnote{This \emph{does not} make the state
                inconsistent. It simply serves to warn the user that the last
                writes might have been lost, e.g.\ the cache didn't flush.}
            \item [$\texttt{02}_{16}$] The disk is in an inconsistent
                state\footnote{This typically only happen if the user poweroffs
                his or her computer while reformatting the disk.}
        \end{description}

        Any other value is considered invalid.

    \section{Virtual device stack (byte 64-504)}
        This stores the virtual device (``vdev'') configuration stack, in the
        order in which the vdev transformations are applied.

        It is a prefix-free code with each codeword starting with an 16-bit
        little-endian label defining the type of the particular vdev.

        The vdevs are specified in~\ref{vdev}.

    \section{Integrity checking (byte 504-512)}
        \subsection{Checksum (byte 504-512)}
        This field stores a little-endian integer equal to the checksum of the
        state block preceding the checksum itself, calculated by the algorithm
        specified in~\ref{config:checksum}.

    \chapter{Virtual devices}
    \label{vdev}
        \section{Terminator}
        This vdev has label 0 and acts as the last applied vdev. No vdevs
        follows.

        \section{Mirror}
        This vdev has label 1. It halfs the parent vdev and lets the higher
        half mirror the lower. Any write to the disk transforms to two
        identical writes into the lower and higher part respectively.

        \section{SPECK encryption}
        This vdev has label 2 and encrypts everything after the disk header
        (\ref{header}) in the parent vdev with SPECK-128 (XEX mode) with
        scrypt key stretching.

        For details, refer to~\ref{algorithm:speck}.

        %%%%%%%%%%%%%%%%%%%%%%%%%%%%%%%%%%%%%%%%%%%%%%%%%%%%%%%%%%%%%%%%%%%%%%%

        \section{Implementation defined}
        This vdev has label $\texttt{ffff}_{16}$ and 8 bytes following, uniquely
        identifying the type\footnote{It is strongly recommended that this
        integer is chosen at random to avoid colliding with other
        implementations.}. The behavior of this vdev and the number of bytes
        following the label and sublevel is entirely left to the implementation
        defined.

    \chapter{State block}
    \label{stateblock}
    The first ``virtual sector'' (the sector following the disk header)
    contains the state block as defined below.

    \section{Integrity checking (0-8)}
        \subsection{Checksum (byte 0-8)}
        This field stores a little-endian integer equal to the checksum of the
        state block following the checksum itself\footnote{This does not have
        the self-validation problem since it is the top block, and silent
        phantom writes won't affect the correctness of the state.}, calculated
        by the algorithm specified in~\ref{config:checksum}.

    \section{Configuration (byte 8-16)}
        \subsection{Compression algorithm (byte 8-10)}
        \label{config:compression}
        This field stores a number in little-endian defining compression
        algorithm in use.

        \begin{description}
            \item [$0$] No compression (identity function).
            \item [$1$] The LZ4 compressor as described
                in~\ref{algorithm:lz4}.
            \item [$\geq 2^{15}$] Implementation defined.
        \end{description}

    \section{State (byte 16-48)}
        \subsection{Super-page pointer (byte 16-32)}
        This field stores some number (in little-endian), which takes values

        \begin{description}
            \item [$n = 0$]    super-page uninitialized.
            \item [$n \neq 0$] $n$ is a pointer (\ref{cluster:page}) to the
                superpage, defined in~\ref{fs:superpage}.
        \end{description}

        \subsection{Freelist head pointer (byte 32-40)}
        \label{state:freelist_head}
        This field stores a cluster pointer (as specified in~\ref{cluster:ptr})
        to some metacluster, subject to the format specified
        in~\ref{cluster:metacluster}.

        \subsection{Checksum of the freelist head (byte 40-48)}
        This field stores the checksum (in little-endian) of the freelist
        head, by the algorithm specified in~\ref{config:checksum}.

    \chapter{Cluster management}

    \section{Clusters and pages}
        The disk is divided into clusters of \clustersize bytes each.

        \subsection{Cluster pointers}
        \label{cluster:ptr}
        Cluster pointers are written as 64-bit little-endian integers, and
        enumerates the sectors, starting at the first non-disk header sector.
        As such, the pointer 0 points to the state block (\ref{stateblock}) and
        can thus be considered a trap value (null pointer).

        \subsection{Pages}
        \label{cluster:page}
        A data cluster contain some number of \clustersize bytes blocks called
        ``pages''.

        A pointer to a page is exactly 128 bits wide (from more less
        significant bits to more significant):

        \begin{description}
            \item [Cluster pointer] This is the cluster pointer in the format
                described in~\ref{cluster:ptr}.
            \item [Little-endian 32-bit page number] If this number is $2^{32}
                - 1$, the page can be read directly from the cluster without
                any specialized logic.  Otherwise, this is the offset (in
                pages) into the decompressed cluster which is compressed by the
                algorithm specified in~\ref{cluster:compression}.
            \item [Little-endian 32-bit checksum] This is the 32 least
                significant bits of the checksum of the page, by the algorithm
                specified in~\ref{config:checksum}.
        \end{description}

        Allocation is done in implementation defined manner.

        \subsection{Meta-cluster format}
        \label{cluster:metacluster}
        Metaclusters are a collection of other free clusters.

        It starts with an 64-bit little-endian checksum of the next
        metacluster. If no next metacluster exists, this integer is 0. This is
        followed by a cluster pointer (\ref{cluster:ptr}) to the next
        metacluster. Similarly, this is 0 if no next metacluster exists.

        Following this, there is some number of cluster points to other free
        clusters. If the metacluster is the top metacluster
        (\ref{state:freelist_head}), the pointers up to the cluster counter
        (\ref{state:freelist_head_counter}). Otherwise, all the pointers from
        the first in this section up to the first null pointer must be free.

        The checksum of a metacluster is calculated
        through~\ref{config:checksum}.

        \subsection{Allocation and deallocation}
        The algorithm for allocation and deallocation is implementation
        defined\footnote{It is generally done by inspecting the head of the
        freelist before popping it, to see if it has a sister cluster,
        where the page can be fit in. If it cannot, the freelist is popped.
        The way such clusters are paired is up to the implementation.
        Bijective maps are recommended for optimal performance.}.

    \section{Compression}
    \label{cluster:compression}
        Data is compressed into fixed size blocks via the algorithm chosen
        in~\ref{config:compression}, but special padding logic is used.

        The compressed data might be padded with $\texttt{ff}_{16}$ and then
        zeros until the desired length is obtained.

    \chapter{Algorithms}

    \section{Checksums}
        \subsection{SeaHash}
        \label{algorithm:seahash}
        SeaHash's initial state is

        \begin{align*}
            a &= \texttt{16f11fe89b0d677c}_{16} \\
            b &= \texttt{b480a793d8e6c86c}_{16} \\
            c &= \texttt{6fe2e5aaf078ebc9}_{16} \\
            d &= \texttt{14f994a4c5259381}_{16}
        \end{align*}

        The input is broken into 64-bit little-endian integers. If necessary,
        the last block is padded with zeros to fit.

        Call this integer $n$, then the updated state is

        \begin{align*}
            a' &= b \\
            b' &= c \\
            c' &= d \\
            d' &= f(a \oplus n)
        \end{align*}

        with $f(n)$ defined by

        \begin{align*}
            p      &=      \texttt{6eed0e9da4d94a4f}_{16} \\
            f_1(x) &\equiv px \pmod{2^{64}} \\
            f_2(x) &=      x \oplus ((x \gg 32) \gg (x \gg 60)) \\
            f(n)   &=      f_1(f_2(f_1(x))))
        \end{align*}

        The final hash value is then produced by

        $$h = f(a \oplus b \oplus c \oplus d \oplus l) $$

        where $l$ is the original length of the (unpadded) hashed buffer.

    \section{Compression algorithms}
        \subsection{LZ4}
        \label{algorithm:lz4}
        LZ4 compressed data is a series of blocks, subject to following format:

        \begin{description}
            \item [1 byte token] Call the higher 4 bits $t_1$ and the lower
                $t_2$.
            \item [$n_1$ 255s (skip if $t_1 \neq 15$)]
            \item [1 byte (skip if $t_1 \neq 15$)] Call this value $e_1$.
            \item [$t_1 + 256n_1 + e_1$ bytes] This (called the literals
                section) is copied directly to the output buffer without any
                pre processing.
            \item [16-bit little-endian integer] Call this value $o$.
            \item [$n_2$ 255s (skip if $t_2 \neq 15$)]
            \item [1 byte (skip if $t_2 \neq 15$)] Call this value $e_1$.
        \end{description}

        After the literals section has been copied to the output buffer, assume
        that the output buffer is now of length $l$ bytes. Then, the bytes from
        $l - O$ to $l - O + t_2 + 256n_2 + e_2$ in the decoded buffer is
        appended to the output stream itself.

        The last block in a stream can be ended after literals section, such
        that no duplicates part is needed.

    \section{Cryptography}
        \subsection{SPECK}
        \label{algorithm:speck}
        This algorithm uses the 128 block size version of the SPECK
        cipher\cite{speck} with the key generated by \cite{scrypt} with $N =
        2^{20}$, $r = 8$, and $p = 1$. \ref{uid} is used as the salt to scrypt.

        The cipher is used in the XEX mode of operation\cite{xex}.

    \begin{thebibliography}{9}
        \bibitem{speck}
        R. Beaulieu, D. Shors, J. Smith, The SIMON and SPECK lightweight block ciphers
        \bibitem{scrypt}
        C. Percival, Stronger Key Derivation Via Sequential Memory-Hard Functions
        \bibitem{xex}
        P. Rogaway, Efficient Instantiations of Tweakable Blockciphers and Refinements to Modes OCB and PMAC
    \end{thebibliography}
\end{document}
