\documentclass[11pt,a4paper]{report}

\usepackage{amsmath,amsfonts,amssymb,hyperref}

\title{Specification of the On-Disk Format of The TFS File System}
\author{TFS team}
\date{\today}

% Constants
\newcommand{\versionnumber}{0 }
\newcommand{\clustersize}{4104 }
\newcommand{\pagesize}{4096 }

\begin{document}
    \maketitle
    \begin{abstract}
        We give a complete specification of the on-disk representation of the
        TFS file system. Certain implementation details are covered.
    \end{abstract}

    \chapter{Disk IO}

    \section{Disk header}
        The first \clustersize bytes are reserved for the ``disk header'' which
        contains configuration and information about the state.

        \subsection{Magic number (0-8)}
        The first 8 bytes are reserved for a magic number, which is used for
        determining if it does indeed store TFS. It is specified to store the
        string ``\texttt{TFS fmt }'' (note the space) in ASCII.

        If the format does only partially conform to this specification, the
        string ``\texttt{\textasciitilde TFS fmt}'' is used instead.

        \subsection{Version number (8-12)}
        Byte 8 to 12 stores a version number, in little-endian. By this
        revision, said number is \versionnumber.

        \subsection{Implementation ID (12-20)}
        Byte 12 to 20 stores some UTF-8 sequence specifying what implementation
        it was last read/written with. By setting this field, the
        implementation explicitly states that it was capable of fully or
        partially reading the image.

        \subsection{Checksum algorithm (32-34)}
        \label{config:checksum}
        Byte 32 to 34 stores a number in little-endian defining checksum
        algorithm in use.

        \begin{description}
            \item [$0$] No checksum (calculated checksum always 0).
            \item [$1$] Fixed checksum as described ~\ref{checksum:fixed}.
            \item [$2$] The SeaHash algorithm as described in
                ~\ref{checksum:seahash}
            \item [$\geq 2^{15}$] Implementation defined.
        \end{description}

        \subsection{Compression algorithm (34-36)}
        \label{config:compression}
        Byte 34 to 36 stores a number in little-endian defining compression
        algorithm in use.

        \begin{description}
            \item [$0$] No compression (identity function).
            \item [$1$] The LZ4 compressor as described in ~\ref{compression:lz4}.
            \item [$\geq 2^{15}$] Implementation defined.
        \end{description}

        \subsection{Encryption algorithm (36-40)}
        \label{config:encryption}
        Byte 36 to 40 stores a number in little-endian defining encryption
        algorithm in use.

        \begin{description}
            \item [$0$] No encryption (identity function).
            \item [$1$] ChaCha20/Poly1305 as described in
                \href{https://tools.ietf.org/html/rfc4253}{RFC \#4253}.
            \item [$\geq 2^{15}$] Implementation defined.
        \end{description}

        The exact semantics of encryption is described in ~\ref{encryption}

        \subsection{Implementation defined (256-512)}
        Byte 256 to 512 are implementation defined configuration values.

        \subsection{Head freelist pointer (512-520)}
        Byte 512 to 520 stores some number (in little endian), which takes
        values

        \begin{description}
            \item [$n = 0$]    no free, allocatable cluster.
            \item [$n \neq 0$] the $n$'th cluster is free and conforms to
                ~\ref{cluster:freelist}.
        \end{description}

        \subsection{Super-page pointer (520-528)}
        Byte 520 to 528 stores some number (in little endian), which takes
        values

        \begin{description}
            \item [$n = 0$]    super-page uninitialized.
            \item [$n \neq 0$] the $n$'th page is the super-page as defined in
                ~\ref{fs:superpage}.
        \end{description}

        \subsection{Implementation defined (1024-2048)}
        Byte 1024 to 2048 are implementation defined state values.

        \subsection{Redundant duplicate (2048-4096)}
        Byte 2048-4096 of the disk header stores a copy of byte 0-2048 of the
        header.

        \subsection{Padding (4096-\clustersize)}
        Byte 4096 to \clustersize is used as padding to achieve the cluster
        size. It is specified to be zeros.

    \section{Clusters and pages}
        The disk is divided into clusters of \clustersize bytes each.

        \subsection{Cluster format}
        Each cluster has a header of 8 bytes:

        \begin{description}
            \item [32-bit checksum] This is the 32 last bits of the
                little-endian checksum of the cluster (algorithm chosen in
                ~\ref{config:checksum}). The first 4 bytes of the cluster are
                not included in the checksum.
            \item [1 byte page flags] Defines which pages the cluster contains.
                If a bit flag is set, the respective page is contained (as
                compressed) in the cluster. If no flags are set, the cluster is
                uncompressed (i.e. represents the page without any special
                representation).
            \item [3 bytes for implementation defined usage] This is
                left to the implementation\footnote{Our implementation
                uses it as garbage collection flag for mark-and-sweep.}.
        \end{description}

        This is followed by the number of set page flags in pages compressed.
        The pages (which are \pagesize byte buffers) are concatenated and
        compressed by the algorithm defined by ~\ref{config:compression}.

        The $n$'th page in some cluster is found by counting the number of set
        flags in the first $n$ bits of the page flags. Call this value $p$,
        then the page is defined as the bytes from $\pagesize p$ to $\pagesize
        (p + 1)$ of the decompressed cluster.

        A cluster can contain up to 8 pages, and the pages are enumerated
        similarly to clusters. The first 61 bits defines what cluster the page
        is stored in. The last 3 bits defines the index of the page in the
        cluster.

        Allocation is done by inspecting the head of the freelist before
        popping it, to see if it has a sister cluster, where the page can be
        fit in. If it cannot, the freelist is popped. The way such clusters are
        paired is implementation defined\footnote{Bijective maps are
        recommended for optimal performance.}.

        \subsection{Cluster freelist}
        \label{cluster:freelist}
        Free clusters has a separate format. It contains a single 64-bit
        little-endian integer, defining the next free cluster. This integer is
        repeated once again to ensure integrity.

        The end of the list is marked by a null pointer (the disk header isn't
        a valid cluster).

        Although not a requirement, it is generally recommended that
        the freelist is kept as monotone as possible\footnote{That is,
        sequential allocations should be local as often as possible in
        order to improve compression ratio.}.

    \section{Checksums}
        \subsection{Fixed}
        \label{checksum:fixed}
        Fixed checksums are independent of the data stored. Instead, the write
        the number of the cluster it is stored in. If data is corrupted, the
        checksum is often invalidated as well, and will thus mismatch.

        \subsection{SeaHash}
        \label{checksum:seahash}
        SeaHash's initial state is

        \begin{align*}
            a &= \texttt{16f11fe89b0d677c}_{16} \\
            b &= \texttt{b480a793d8e6c86c}_{16} \\
            c &= \texttt{6fe2e5aaf078ebc9}_{16} \\
            d &= \texttt{14f994a4c5259381}_{16}
        \end{align*}

        The input is broken into chunks of 32 bytes, or 4 64-bit little-endian
        integers. Call these integers $(p, q, r, s)$ respectively. Then
        updating state is defined by

        \begin{align*}
            a' &\equiv f(a + p) \pmod{2^{64}} \\
            b' &\equiv f(b + q) \pmod{2^{64}} \\
            c' &\equiv f(c + r) \pmod{2^{64}} \\
            d' &\equiv f(d + s) \pmod{2^{64}}
        \end{align*}

        with $f(n)$ defined by

        \begin{align*}
            k      &=      \texttt{7ed0e9fa0d94a33}_{16} \\
            f_1(n) &=      n \oplus (x \gg 32) \\
            f_2(n) &\equiv kn \pmod{2^{64}} \\
            f(n)   &=      f_1(f_2(f_1(f_2(f_1(n)))))
        \end{align*}

        The final hash value is then produced by

        $$h \equiv a + f(b) + f(c + f(d)) \pmod{2^{64}}$$

        If a byte, $e$, is excessive (i.e. the length is not divisible by 32), it is included through

        $$h' \equiv f(h + e) \pmod{2^{64}}$$

        This process is repeated, until no more excessive bytes are left.

    \section{Compression algorithms}
        \subsection{LZ4}
        \label{compression:lz4}
        LZ4 compressed data is a series of blocks, subject to following format:

        \begin{description}
            \item [1 byte token] Call the higher 4 bits $t_1$ and the lower $t_2$.
            \item [$n_1$ 255s (skip if $t_1 \neq 15$)]
            \item [1 byte (skip if $t_1 \neq 15$)] Call this value $e_1$.
            \item [$t_1 + 256n_1 + e_1$ bytes] This (called the literals
                section) is copied directly to the output buffer without any
                pre processing.
            \item [16-bit little-endian integer] Call this value $o$.
            \item [$n_2$ 255s (skip if $t_2 \neq 15$)]
            \item [1 byte (skip if $t_2 \neq 15$)] Call this value $e_1$.
        \end{description}

        After the literals section has been copied to the output buffer, assume
        that the output buffer is now of length $l$ bytes. Then, the bytes from
        $l - O$ to $l - O + t_2 + 256n_2 + e_2$ in the decoded buffer is
        appended to the output stream itself.

        The last block in a stream can be ended after literals
        section, such that no duplicates part is needed.

    \section{Encryption}
    \label{encryption}
        If enabled (see ~\ref{config:encryption}), every cluster is encrypted
        separately.

        Encrypting cluster $n$ is done by obtaining a
        key\footnote{Equivalently, prepending the 64-bit integer defining which
        cluster is being encrypted, to some user-specified key.} $k' = 2^{64}k
        + n$ and encrypting the cluster with said key by the algorithm given in
        ~\ref{config:encryption}.
\end{document}
