\documentclass[11pt,a4paper]{report}

\usepackage{amsmath,amsfonts,amssymb}

\title{Specification of the On-Disk Format of The TFS File System}
\author{TFS team}
\date{\today}

% Specify the version number for this revision.
\newcommand{\versionnumber}{0}

\begin{document}
    \maketitle
    \begin{abstract}
        We give a complete specification of the on-disk representation of the
        TFS file system. Certain implementation details are covered.
    \end{abstract}

    \chapter{Disk IO}

    \section{Disk header}
        The first 4096 bytes are reserved for the ``disk header'' which
        contains configuration and information about the state.

        \subsection{Magic number (0-8)}
        The first 8 bytes are reserved for a magic number, which is used for
        determining if it does indeed store TFS. It is specified to store

        $$5446532064617461_{16}$$

        in little-endian.

        \subsection{Version number (8-12)}
        Byte 8 to 12 stores a version number, in little-endian. By this
        revision, said number is \versionnumber.

        \subsection{Checksum algorithm (32-34)}
        \label{config:checksum}
        Byte 32 to 34 stores a number in little-endian defining checksum
        algorithm in use.

        \begin{description}
            \item [$0$] The SeaHash algorithm as described in
                ~\ref{checksum:seahash}
            \item [$\geq 2^{15}$] Implementation defined.
        \end{description}

        \subsection{Compression algorithm (34-36)}
        \label{config:compression}
        Byte 34 to 36 stores a number in little-endian defining compression
        algorithm in use.

        \begin{description}
            \item [$0$] The LZ4 compressor as described in
                ~\ref{compression:lz4}.
            \item [$\geq 2^{15}$] Implementation defined.
        \end{description}

        \subsection{Implementation defined (36-512)}
        Byte 36 to 512 are implementation defined configuration values.

        \subsection{Head freelist pointer (512-520)}
        Byte 512 to 520 stores some number (in little endian), which takes
        values

        \begin{description}
            \item [$n = 0$]    no free, allocatable cluster.
            \item [$n \neq 0$] the $n$'th cluster is free and conforms to
                ~\ref{cluster:freelist}.
        \end{description}

        \subsection{Super-page pointer (520-528)}
        Byte 520 to 528 stores some number (in little endian), which takes
        values

        \begin{description}
            \item [$n = 0$]    super-page uninitialized.
            \item [$n \neq 0$] the $n$'th page is the super-page as defined in
                ~\ref{fs:superpage}.
        \end{description}

        \subsection{Implementation defined (1024-2048)}
        Byte 1024 to 2048 are implementation defined state values.

        \subsection{Redundant duplicate (2048-4096)}
        The last half of the disk header stores a byte-by-byte duplicate of the
        first half.

    \section{Clusters and pages}
        The disk is divided into clusters of 4100 bytes each.

        \subsection{Cluster format}
        Each cluster has a header of 5 bytes:

        \begin{description}
            \item [31 bit checksum] This is the 31 last bits of the checksum of
                the cluster (algorithm chosen in ~\ref{config:checksum}). The
                first 4 bytes are not included in the checksum.
            \item [1 bit of implementation defined usage] This is left to the
                implementation \footnote{Our implementation uses it as garbage
                collection flag for mark-and-sweep.}
            \item [1 byte page flags] Defines which pages the cluster contains.
                If a bit flag is set, the respective page is contained (as
                compressed) in the cluster. If no flags are set, the cluster is
                uncompressed (i.e. represents the page without any special
                representation).
        \end{description}

        This is followed by the number of set page flags in pages compressed.
        The pages (which are 4096 byte buffers) are concatenated and compressed
        by the algorithm defined by ~\ref{config:compression}.

        The $n$'th page in some cluster is found by counting the number of set
        flags in the first $n$ bits of the page flags. Call this value $p$,
        then the page is defined as the bytes from $4096p$ to $4096(p + 1)$ of
        the decompressed cluster.

        A cluster can contain up to 8 pages, and the pages are enumerated
        similarly to clusters. The first 61 bits defines what cluster the page
        is stored in. The last 3 bits defines the index of the page in the
        cluster.

        Allocation is done by inspecting the head of the freelist before
        popping it, to see if it has a sister cluster, where the page can be
        fit in. If it cannot, the freelist is popped. The way such clusters are
        paired is implementation defined\footnote{Bijective maps are
        recommended for optimal performance.}.

        \subsection{Cluster freelist}
        \label{cluster:freelist}
        Free clusters has a separate format. It contains a single 64-bit
        little-endian integer, defining the next free cluster. This integer is
        repeated once again to ensure integrity.

        The end of the list is marked by a null pointer (the disk header isn't
        a valid cluster).

        Although not a requirement, it is generally recommended that the
        freelist is kept as monotone as possible \footnote{That is,
        sequential allocations should be local as often as possible in order
        to improve compression ratio.}.

    \section{Checksums}
        \subsection{SeaHash}
        \label{checksum:seahash}
        SeaHash's initial state is

        \begin{align*}
            a &= \texttt{16f11fe89b0d677c}_{16} \\
            b &= \texttt{b480a793d8e6c86c}_{16} \\
            c &= \texttt{6fe2e5aaf078ebc9}_{16} \\
            d &= \texttt{14f994a4c5259381}_{16}
        \end{align*}

        The input is broken into chunks of 32 bytes, or 4 64-bit little-endian
        integers. Call these integers $(p, q, r, s)$ respectively. Then
        updating state is defined by

        \begin{align*}
            a' &\equiv f(a + p) \pmod{2^{64}} \\
            b' &\equiv f(b + q) \pmod{2^{64}} \\
            c' &\equiv f(c + r) \pmod{2^{64}} \\
            d' &\equiv f(d + s) \pmod{2^{64}}
        \end{align*}

        with $f(n)$ defined by

        \begin{align*}
            k      &=      \texttt{7ed0e9fa0d94a33}_{16} \\
            f_1(n) &=      n \oplus (x \gg 32) \\
            f_2(n) &\equiv kn \pmod{2^{64}} \\
            f(n)   &=      f_1(f_2(f_1(f_2(f_1(n)))))
        \end{align*}

        The final hash value is then produced by

        $$h \equiv a + f(b) + f(c + f(d)) \pmod{2^{64}}$$

        If a byte, $e$, is excessive (i.e. the length is not divisible by 32), it is included through

        $$h' = f(h + e)$$

    \section{Compression algorithms}
        \subsection{LZ4}
        \label{compression:lz4}
        LZ4 compressed data is a series of blocks, subject to following format:

        \begin{description}
            \item [1 byte token] Call the higher 4 bits $t_1$ and the lower $t_2$.
            \item [$n_1$ 255s (skip if $t_1 \neq 15$)]
            \item [1 byte (skip if $t_1 \neq 15$)] Call this value $e_1$.
            \item [$t_1 + 256n_1 + e_1$ bytes] This (called the literals
                section) is copied directly to the output buffer without any
                pre processing.
            \item [16-bit little-endian integer] Call this value $o$.
            \item [$n_2$ 255s (skip if $t_2 \neq 15$)]
            \item [1 byte (skip if $t_2 \neq 15$)] Call this value $e_1$.
        \end{description}

        After the literals section has been copied to the output buffer, assume
        that the output buffer is now of length $l$ bytes. Then, the bytes from
        $l - O$ to $l - O + t_2 + 256n_2 + e_2$ in the decoded buffer is
        appended to the output stream itself.

        The last block in a stream can be ended after $L$, such that no
        duplicates part is needed.
\end{document}
